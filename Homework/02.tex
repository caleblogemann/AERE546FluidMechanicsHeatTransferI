\documentclass[11pt, oneside]{article}
\usepackage[letterpaper, margin=2cm]{geometry}
\usepackage{AERE546}
\usepackage{xspace}

\begin{document}
\noindent \textbf{\Large{Caleb Logemann \\
AER E 546 Fluid Mechanics and Heat Transfer I \\
Homework 2
}}

%\lstinputlisting[language=MATLAB]{H01_23.m}
\begin{enumerate}
  \item % #1
    The heat fin equation is the linear o.d.e.
    \[
      \d[2]{T}{x} = MT
    \]
    where $M$ is a sort of thermal mass.
    First write the finite difference equation in terms of a tridiagonal matrix.
    Solve that equation using the Thomas algorithm (Gaussian elimination) for:
    \begin{enumerate}
      \item[(a)]
        Compute a solution with the boundary conditions $T(0) = 1$ and
        $T(1) = 0$.
        This corresponds to a fin that is between a hot and a cold reservoir.
        In non-dimensional terms, the heat flux into the cold reservoir is
        $-\d{T}{x}$ at $x = 1$.
        Obtain the heat flux as $x = 1$ for $M = 1, 5, 9$. Use enough grid
        points to obtain 1\% accuracy.
        Provide your three numerical values of the heat flux.
        Provide a single graph with curves of $T(x)$ for the 3 values of $M$.



      \item[(b)]
        Compute a solution with the boundary conditions
        $T(0) = 1$, $\d{T(1)}{x} = 0$.
        This corresponds to a fin that is insulated at one end.
        Solve for the temperature, $T(1)$, at the insulated end for
        $M = 1, 5, 9$.
        Provide your three numerical values of $T(1)$.
        Also plot $T(x)$ for $M = 9$ with each pair of boundary conditions and
        compare to the exact solution.

      \item[(c)]
        Add a distributed heat source: Compute and plot a solution of the
        non-homogeneous equation
        \[
          \d[2]{T}{x} = MT - 100 x^2 \p{1 - x}^2
        \]
        with $M = 9$, $T(0) = 1$ and $\d{T(1)}{x} = 0$.
    \end{enumerate}

  \item % #2
    \begin{enumerate}
      \item[(i)]
        What type of p.d.e.\ is
        \[
          \mpd[2]{\phi}{\partial x \partial y} + \phi = 25?
        \]

      \item[(ii)]
        What type of p.d.e.\ does the velocity potential, $\phi$, satisfy if
        \[
          \pd{u}{x} + \pd{v}{y} = 0
        \]
        with
        \[
          u = \pd{\phi}{x} \qquad v = \pd{\phi}{y}?
        \]

      \item[(iii)]
        The boundary layer momentum equation is
        \[
          u \pd{u}{x} + v \pd{u}{y} = \frac{1}{Re} \pd[2]{u}{y}
        \]
        where $Re$ is the Reynolds number.
        What type is this equation?
    \end{enumerate}
\end{enumerate}
\end{document}

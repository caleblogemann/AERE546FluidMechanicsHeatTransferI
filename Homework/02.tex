\documentclass[11pt, oneside]{article}
\usepackage[letterpaper, margin=2cm]{geometry}
\usepackage{AERE546}
\usepackage{xspace}

\begin{document}
\noindent \textbf{\Large{Caleb Logemann \\
AER E 546 Fluid Mechanics and Heat Transfer I \\
Homework 2
}}

%\lstinputlisting[language=MATLAB]{H01_23.m}
\begin{enumerate}
  \item % #1
    \begin{enumerate}
      \item[(a)]

        First I will establish a some notation.
        I will discretize the fin into $N + 1$ points.
        Let $x_i = \frac{i}{N}$, then $x_0 = 0$ and $x_N = 1$.
        Let the approximate solution at $x_i$ be represented by $T_i$.
        Then a numerical solution consists of a set of values $T_i$ for
        $i \in \NN$, $0 \le i \le N$.

        Next I will discretize the partial differential equation into a discrete
        equation.
        The second order central finite difference for the second derivative is
        \[
          \frac{T_{i - 1} - 2T_i + T_{i+1}}{\Delta x^2}
        \]
        Plugging this into the partial differential equation gives the following
        difference equation
        \[
          \frac{T_{i - 1} - 2T_i + T_{i+1}}{\Delta x^2} = MT_i
        \]
        Simplifying this gives
        \[
          T_{i - 1} + \p{- 2 - M\Delta x^2}T_i + T_{i+1} = 0
        \]

        For this problem the boundary conditions are $T(0) = 1$ and $T(1) = 0$.
        This can be encoded into the numerical solution at $T_0 = 1$ and
        $T_N = 0$.
        Now only the values for $T_i$ for $1 \le i \le N - 1$ need to be found.
        These can be found by solving the equations
        \[
          T_{i - 1} + \p{- 2 - M\Delta x^2}T_i + T_{i+1} = 0
        \]
        for $i = 1$, this become
        \[
           \p{- 2 - M\Delta x^2}T_1 + T_2 = -1
        \]
        and for $i = N - 1$ the equation is
        \[
          T_{N - 2} + \p{- 2 - M\Delta x^2}T_{N - 1} = 0
        \]
        These equation can be written in matrix form as
        \[
          \begin{bmatrix}
            -2 - \Delta x^2 & 1 & & & 0 \\
            1 & -2 - \Delta x^2 & 1 &  \\
             & \ddots & \ddots & \ddots &  \\
             &  & 1 & -2 - \Delta x^2  & 1  \\
            0 &  &  & 1 & -2 - \Delta x^2
          \end{bmatrix}
          \begin{bmatrix}
            T_1 \\
            T_2 \\
            \vdots \\
            T_{N - 2} \\
            T_{N - 1}
          \end{bmatrix}
          =
          \begin{bmatrix}
            -1 \\
            0 \\
            \vdots \\
            0 \\
            0
          \end{bmatrix}
        \]
        This is a tridiagonal system which can be solved easily with the
        Thomas algorithm.

      \item[(b)]

        For this problem 
      \item[(c)]
    \end{enumerate}
  \item % #2
\end{enumerate}
\end{document}

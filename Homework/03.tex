\documentclass[11pt, oneside]{article}
\usepackage[letterpaper, margin=2cm]{geometry}
\usepackage{AERE546}
\usepackage{xspace}

\begin{document}
\noindent \textbf{\Large{Caleb Logemann \\
AER E 546 Fluid Mechanics and Heat Transfer I \\
Homework 3
}}

%\lstinputlisting[language=MATLAB]{H01_23.m}
\begin{enumerate}
  \item % #1
    A slab of metal is initially at uniform temperature.
    One end is suddenly raised to a high temperature, while the other end is kept cool.
    Compute the penetration of heat into the slab as a function of time.
    In dimensional form the temperature diffusion equation, initial and boundary values are
    \[
      \pd{T^*}{t_*} = \kappa \pd[2]{T^*}{x_*} \qquad T^*(x,0) = 0 \qquad T^*(0, t) = 0, T^*(L, t) = T_w.
    \]
    Non-dimensionalize temperature by $T_w$ and length by $L$ and time by $L^2/\kappa$.
    Integrate by Euler Explicit, up to a non-dimensional time of $0.3$.
    Use $N_x = 121$ grid points in $x$.
    Let $\Delta t = \alpha \Delta x^2$.
    Try a value of $\alpha > 0.5$.
    What happens?
    Why?
    How small must $\alpha$ be to obtain an accurate solution?
    Provide a single figure with line plots of the solution at time intervals of 0.04.
    Note that the computational time-step will be smaller than 0.04.
    The bulk heat transfer coefficient is defined as
    \[
      h_T = \frac{Q}{T(1) - T(0)}
    \]
    where $T = \eval{\pd{T}{x}}{x = 1}{}$ is the heat flux into the slab.
    Plot $h_T$ as a function of time for $t > 0.01$.

  \item % #2
    A slab is heated by shining a laser on it.
    The laser is shut off and the heat diffuses throughout the slab.
    Its ends are insulated.
    This is modeled as the non-dimensional problem: solve
    \[
      \pd{T}{x} = \pd*{\kappa \pd{T}{x}}{x}
    \]
    in the interval $0 \le x \le 1$, with the initial condition
    \[
      T(x, 0) = \frac{e^{-\p{x - 0.5}^2/\sigma^2}}{\sigma \sqrt{\pi}}, \quad \sigma = 0.1
    \]
    The slab length is normalized to unity and $\sigma$ characterizes the
    region heated by the laser.
    Consider a material with variable diffusivity.
    Let
    \[
      \kappa = 0.1 + 2.0 e^{-5x}
    \]
    The no-flux boundary condition
    \[
      \pd{T}{x}(0, t) = 0 = \pd{T}{x}(1, t)
    \]
    is applied at the insulated ends.
    Use second order Runge-Kutta.
    Solve with about $N_x = 250$ grid points in $x$.
    Choose a small time-step to obtain an accurate solution.
    Integrate up to a non-dimensional time of 0.05.
    Provide a single plot containing the initial condition and curves showing
    the solution $T(x)$ at time intervals of 0.01.
    Plot $\dintt{0}{1}{T(x)}{x}$ versus time.
    What should the value of the integral be?

  \item % #3
    Now consider the case where one end of the slab is insulated and the other
    is held at constant temperature:
    \[
      \pd{T}{x}(0) = 0 \quad T(1) = 1
    \]
    solve the constant diffusivity, diffusion equation as in the first problem,
    but use Crank-Nicholson.
    \begin{enumerate}
      \item[(a)]
        Set $\Delta t = \alpha \Delta x^2$.
        Try a couple of relatively large value of $\alpha$ and see whether your
        calculation converges, or blows up (should it?).

      \item[(b)]
        Provide a single figure with plots of $T(x)$ at intervalue of $0.04$ up to $t = 0.4$.
        Explain why your solution makes sense.
        The bulk heat transfer coefficient is defined as
        \[
          h_T = \frac{Q}{T(1) - T(0)}
        \]
        where $T = \eval{\pd{T}{x}}{x = 1}{}$ is the heat flux into the slab.
        Plot $h_T$ as a function of time for $t > 0.01$.
    \end{enumerate}

  \item % #4
    \begin{enumerate}
      \item[(a)]
        Is the scheme
        \[
          U_i^{n+1} = U_i^n - \frac{C}{2} \p{U_{i+1}^n - U_{i-1}^n}
        \]
        stable, conditionally stable or absolutely unstable?
        C is a constant.

      \item[(b)]
        Is the scheme
        \[
          U_i^{n+1} = U_i^n - \frac{C}{2} \p{U_{i+1}^{n+1} - U_{i-1}^{n+1}}
        \]
        stable, conditionally stable or absolutely unstable?
        C is a constant.

      \item[(c)] % Done
        Which method would be called implicit?

        The scheme in (b) would be called implicit as the value of $U^{n+1}_i$
        depends on the values $U^{n+1}_{i+1}$ and $U^{n+1}_{i-1}$.
        The solution for a point at time $t^{n+1}$ depends on the points next
        to it at the same time.
        This means that a system of equations must be solved in order to update
        the solution.
    \end{enumerate}
\end{enumerate}
\end{document}

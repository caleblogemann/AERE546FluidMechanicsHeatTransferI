\documentclass[11pt, oneside]{article}
\usepackage[letterpaper, margin=2cm]{geometry}
\usepackage{AERE546}
\usepackage{xspace}

\begin{document}
\noindent \textbf{\Large{Caleb Logemann \\
AER E 546 Fluid Mechanics and Heat Transfer I \\
Homework 1
}}

%\lstinputlisting[language=MATLAB]{H01_23.m}
\begin{enumerate}
  \item[\#1]
    \item[(a)]
      How many `data' points are needed to obtain a third order accurate
      polynomial approximation?
      Derive a finite difference formula for $\partial T/\partial x$ that is
      third order accurate in $\Delta x$.
      Use only the minimum number of points.


    \item[(b)]
      Derive the second order accurate centered difference formula for
      $\pd[2]{T}{x}$.

  \item[\#2]
    \item[(a)]
      The equation for a damped oscillator is
      \[
        \ddot{Y} + \sigma \dot{Y} + \omega^2 Y = 0.
      \]
      Let the non-dimensional frequency be $\omega = 1$.
      Consider the two damping rates $\sigma=0.0$ and $\sigma=0.5$.
      Solve this by RK2, out to $t = 32$, with the intial conditions $Y(0) = 1$
      and $\dot{Y}(0) = 0$.
      The time-step can be $\Delta t = 32/N$, where $N$ is the number of
      integration points.
      Plot solutions with $N = 21, 101, 301$.
      What is the analytical solution?
      Compare your numerical solutions to the exact result.

      First I will compute the analytical solution to this differential
      equation.
      This can be done by finding the characteristic polynomial of the equation,
      which is
      \[
        r^2 + \sigma r + 1 = 0.
      \]
      Using the quadractic formula, we see that the roots of this polynomial
      are $r = -\frac{\sigma}{2} \pm \frac{\sqrt{\sigma^2 - 4}}{2}$.
      When $\sigma = 0.0$, the roots are $r = \pm i$.
      In the case of complex roots the general solution will be
      \[
        Y(t) = c_1 \cos{t} + c_2 \sin{t}.
      \]
      Using the intial conditions we see that the exact solution is
      \[
        Y(t) = \cos{t}.
      \]

      When $\sigma = 0.5$ the roots are $r = -\frac{1}{4} \pm \frac{\sqrt{15}}{4} i$.
      In this case the general solution is
      \[
        Y(t) = e^{-\frac{1}{4}t} \p{c_1 \cos{\frac{\sqrt{15}}{4} t} + c_2 \sin{\frac{\sqrt{15}}{4} t}}
      \]
      and the exact solution with boundary conditions is
      \[
        Y(t) = e^{-\frac{1}{4}t} \cos{\frac{\sqrt{15}}{4} t}.
      \]

      Now in order to solve this equation numerically with RK2, we first need to
      transform this second order differential equation into a system of first
      order differential equations.
      To do this let $Z = \dot{Y}$, then the system becomes
      \begin{align*}
        \dot{Y} &= Z \\
        \dot{Z} &= -\sigma Z - \omega^2 Y
      \end{align*}
      This is in the form $\dot{x} = RHS(x)$ where
      \begin{align*}
        x &= \br{Y, Z}^T \\
        RHS(x) &= br{x_2, -\sigma x_2 - \omega^2 x_1}^T.
      \end{align*}



    \item[(b)]
      The equation for a nonlinear spring (without damping) is
      \[
        \ddot{Y} + Y - BY^3 = 0.
      \]
      Solve by RK2 out to $t = 32$ with the intial conditions $Y(0) = 1$ and
      $\dot{Y}(0) = 0$.
      Plot $Y(t)$ for $B = 0.2, 0.6, 0.9, 0.999$.
      Chose $N$ large enough to get an accurate solution; that will depend on
      the value of $B$.

  \item[\#3]
    Repeat the linear spring computation (ex. 2.a) with AB2.
    What does the solution for $\sigma=0.0$ tell you about the stability of AB2?
\end{enumerate}
\end{document}
